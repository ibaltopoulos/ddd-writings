\documentclass{article}

\begin{document}
\title{Build pipeline}

\maketitle

\subsection*{How we use nant}

\begin{itemize}
\item Each nant file contains: external and internal targets. The
  later have a dash (-) in front of their name.
\item Each nant file is responsible for a particular stage of the
  pipeline.
\item We try to use explicit call targets instead of the declarative
  depends.
\end{itemize}



\subsection*{Implementing continuous integration}

What do I need to implement a continuous integration and continuous
delivery system?  

\begin{itemize}
\item \textbf{Version control} What should I put in version control?
  Everything required by your project: code, tests, database scripts,
  build and deployment scripts and anything else needed to create,
  install, run, and test the applicatin.
\item \textbf{An automated build} You must be able to start your build
  from the command line. Whatever the mechanism, it must be possible
  for either a human or a computer to run the build, test, and
  deployment process in an automated fashion via the command line.
\item \textbf{Agreement of the team} Continuous integration is a
  practice not a tool.
\end{itemize}


\subsection*{Some principles}

\begin{itemize}
\item To enable version control of the pipeline scripts we want to
  have as much of the scripting logic inside the nant script files and
  as little as possible inside the Continuous Integration server (TC).
\end{itemize}


\subsection*{Essential practices}
\begin{itemize}
  \item Don't check in on a broken build
  \item Always run all your commit tests locally before commiting
  \item Wait for commit tests to pass before moving on
  \item Never go home on a broken build
  \item Always be prepared to revert to the previous revision
  \item Time-box fixing before reverting your changes
  \item Don't comment out failing tests
  \item Take responsibility for all breakages that result from your
    changes
\end{itemize}

\subsection*{Principles of Software delivery}

\begin{itemize}
  \item Create a repeatable, reliable process for releasing software
  \item Automate almost everything
  \item Keep everything in version control
  \item If it hurts, do it more frequently, and bring the pain forward
  \item build quality in
  \item Done means released
  \item Everybody is responsible for the delivery process.
\end{itemize}

\subsection*{Principles of configuration management}

\begin{itemize}
\item Keep absolutely everything in version control
\item Check in regularly to trunk
\item Use meaningful commit messages, every commit message must have
  an associated Jira issue number unless it's a Merge. No exceptions!
\end{itemize}


\end{document}