\documentclass{article}

\begin{document}
\title{Agile development in a financial services startup: How ``code first''
helps us release 15 times a day}
\date{October 25, 2011}

\maketitle


\begin{abstract}
Orbis is a global investment firm managing USD25B for institutional
clients and high net-worth investors.
%
Our vision is to 'democratize' high quality investments by offering
them direct to everyday investors, through our new online platform
OrbisAccess.
%
In this talk we go over our journey so far to build a high-tech
startup from the ground up.
%
We reflect on how domain driven design (DDD) and ``code first'' have
helped us keep our core model clean from technology distractions,
create testable code, and remain flexible in light of business
requirement changes.
%
Finally we share our experience using continuous delivery to quickly
evolve functionality, reduce the risk of deploying to production, and
get real-time feedback on the progress of the entire project.
\end{abstract}




\paragraph{Speaker bio} 
Alvin Mok is the global manager for OrbisAccess. 
%
His current focus is to incubate an internal start-up which will
revolutionize the financial industry with state-of-the-art technology,
novel business model and innovative products.
%
Prior to joining Orbis, Alvin was an Engagement Manager at McKinsey
Hong Kong, where he was actively involved in the high tech and
infrastructure practices as well as co-founded the Greater China real
estate practice.
%
Earlier in his career, Alvin was also a venture capitalist, held
positions in Microsoft, and founded his own startup.
%
Alvin holds a BASc from University of Toronto where he received the
Governor General Award in the Engineering Science program and a MBA
from Harvard Business School where he was a Baker Scholar.

\paragraph{Speaker bio}
Ioannis Baltopoulos is the back-end developer lead for Orbis Access.
%
Prior to Orbis, Ioannis was studying for his PhD degree at the
University of Cambridge.
%
He has worked at Microsoft Research Cambridge, Sun Microsystems
Research Labs in Silicon Valley and CERN in Geneva on a variety of
areas ranging from web services for distributed computing to type
systems in multi-tier language compilers.
%
Ioannis holds a BSc degree from the University of Kent in Computer
Science and a MSc degree in Advanced Computing from Imperial College.

\paragraph{Speaker bio}
Paul Coleman is a software developer for Orbis Investment Management
Limited.  He is responsible for parts of the back end, business logic
and data layers in the OrbisAccess system.
%
Prior to Orbis, Paul worked for Logic Communications as a Vice President
of Marketing, responsible for the market strategy for all of Logic's
Business and Residential products and services.
%
Paul also worked for Microsoft and was responsible for the user
interface implementation of Office, technical and international
competition strategy, and delivering technical presentations to
executives from around the globe.
%
Paul holds an MBA degree from the University of Washington and a
computer science undergraduate degree with honors from Cornell
University.  Paul is also a technology advisor to the Bermuda
Government and Vice President of Bermuda's .NET User Group.
%
Paul is Bermudian and lives in Bermuda with his wife and two young children.

\paragraph{Speaker bio} 
Qi Chen works as a back-end developer in the OrbisAccess London team
where he's responsible for business logic implementation and database
deployment automation.
%
He graduated with a first class MEng Computing degree from Imperial
College.
%
Prior to that, he did a half year placement for IBM in their network
management software development team where he was responsible for
software accessibility testing and management, automatic deploying and
testing framework development, as well as assisting project
management.

\tableofcontents

\section{Structure}
\subsection{Intro (10 mins, Paul?)}
I think Orbis has an exceptionally good back story that fits perfectly
with his first and second points.  Additionally it sets us apart from
the rest of the financial services industry.  It's the Allan Gray
story as motivation for why we are doing what we are doing.  The crux
of the story is that he's had the vision and inclination for a long
time but for retail to be successful the cost of business needs to be
lower and we are using technology to do so.

The only downside I see about this as the opening / motivation is that
planning for retirement is probably not on the minds of the students.
However we can draw them in by asking about their parents, and if they
plan to live long enough to get to their parent's age

\subsection{OA Operations Areas (10 mins, Alvin?)}
The motivation sets up what we are doing and why IT is the single most
important tool for us winning in the retail space.  Then we move into
all the operational aspects of the business - enumerating marketing,
information gathering by customers, account opening, transactions,
funds, customer support.  And at a high level at this point we can
discuss how we are using technology to make the operational area
better for the customer and cheaper for us.

\subsection{Drill down (20 mins, Ioannis, Qi)}
Finally we can drill into one or two of these areas to explain in
detail what technology choices we've made and why and pros, the cons,
the impact to schedule, the skills on the team we need to be
successful in that area etc.  What areas we need to drill down on
would still be tbd.

\subsection{Closing (5 mins, ?)}

\subsection{Questions (12 mins, Ioannis, Qi, Paul, Alvin)}

\subsection{Goodbye (3 mins, Alvin)}
 
\pagebreak

\section{Points}

\paragraph{Guiding Agile Principles}
In preparing these slides I was trying to come up with an inspiring
opening that would grab everyone's attention.
%
So as I was searching around the web for inspiration I remembered the
principles behind the Agile Manifesto which, I must admit, hadn't
looked at since I took Susan EisenBach's Advanced Topics in Software
Engineering course a few years ago.
%
As I was reading them I slowly realised how relevant they are to our
technology team and I will explain how in a moment.
%
So if you forget everything else I say today at least remember to go
read those principles. These principles should be the heart and soul
of any software development team.
%
It takes 5 minutes to read them, but much longer for your entire team
to master them.

Let me highlight some items on that list.

\begin{itemize}
\item Our highest priority is to satisfy the customer through early
  delivery of valuable software

\item Build projects around motivated individuals.  Give them the
  environment and support they need, and trust them to get the job
  done.

\item Simplicity --the art of maximizing the amount of work not done--
  is essential.
\end{itemize}

\paragraph{Great but we're a startup}
\begin{itemize}
\item Satisfy the customer, which customer? we don't even know who to
  target yet!

\item Where are all of these motivated individuals hidding?

\item Easier said than done when you have the FSA.
\end{itemize}


\paragraph{What matters when developing code in a startup
  environment?}
\begin{itemize}
\item the software artefact must always work

\item the time between inception of the idea until we have a working
  version of it must be minimal

\item small incremental changes are preferred to big bang changes

\item business functionality is preferred over framework code

\item good programmers are hard to find and unfortunately they don't
  show up when you need them

\item your business team is also trying to make sense of this all so
  they don't have the answers for you. You need to make an educated
  guess and write coded flexible enough to anticipate the changes.

\item  you are always faced with the choice: should I just fix this
  now or should I write an automated script to fix it?
\end{itemize}


\paragraph{Additionally being a web startup}
\begin{itemize}
\item  your users expect you to be able to be able to quickly fix the
website

\item  you don't need release cycles like desktop software. Delivery
  of your software is done automatically for you. Even if it's a
  large application there are ways to deliver it in pieces,
  progressive JavaScript downloading, requests from multiple
  hosts with different names, etc

\item  it's easy to fake a demo. Midway through you switch from the
  production URl to localhost. Alvin you don't hear that from me.
\end{itemize}



\section{Working software is the primary measure of progress}

The waterfall model is a good way to develop software if you know
every requirement upfront.
%
If you are lucky enough to be in such a situation the model has clear
steps of analysis, design, implementation, testing and delivery.
%
But in a startup environment you don't know all the requirements up
front.
%
The business team is working alongside the development team trying to
clarify their ideas and what and articulate their vision.
%
In such and environment, you development team still needs to make
progress and create working software.

Typically what happens at this point is that the development team
disappears for a couple of months working on the ``database layer'' or
evaluating the architecture.
%
When they come back from ``intellectual curiocity land'' the business
team presents them with some simple feature, like a login page, and
they take an huge amount of time because the framework didn't consider
that case.


\subsection{``Code first'' and DDD}

The alternative is to focus straight away to the features you knwo
you'll need and leave the infrastructure for later when you've
understood the real requirements.
%
This is what code first and domain driven design is all about.

\subsection{Implementing a simple feature}

The business team requested a feature:
%
Username can be an alphanumeric string
%
They want the ability for the the browser to remember the user's
name on the particular computer.
%
Trying to protect potential users from key-loggers, the third thing
they want, is to be able to ask for specific characters from one's
password.

\subsection{Database first}





\paragraph{Starting development from the database model is inefficient}
\begin{itemize} 
\item for every new concept/table you introduce, you have to create
  data access objects in the object language

\item it gives higher priority to data organization rather than data
  usage
\item your persistence mechanism imposes a relational way of viewing
  the world. For some applications the relational model is not
  appropriate. A hierarchical representation would have been better

\item you don't even know the domain yet, we're discovering it as we
  go along
\item during the ramp up stage of the project there are frequent
  changes to the model. That means constant code churn
\end{itemize}

\paragraph{What is code first or domain driven design?}
- object oriented programming done right

\paragraph{Why is it good?}




\section{Our highest priority is to satisfy the customer through early and continuous delivery of valuable software}

\subsection{Why is continuous delivery so important?}

%
Because, as a browser user it sucks to be asked to get a new browser
when all I want to do is to check my email.
%
Because, as a music lover I hate being forced to restart my computer
when all I want to do is to listen to some music.
%
Because, as a bank customer and a computer scientist, I expect my bank
to be able to update their systems without asking me to acknowledge
their routine maintenance.
%
 
\subsection{So what's the problem with periodic updates?}

They have a huge cost in time and money for all of us!
%
They affect your users and they affect your technology team.

\subsubsection*{How do periodic updates affect the users?}
\begin{itemize}
  \item \textbf{They destroy the user's experience}
%
Users come to your site or use your application either because they
need to, they're trying to achieve a goal (I need to go to my banking
site to pay my bills) or because they take pleasure in what you're
providing them with (I really love listening to music when I code).
%
Any roadblocks between your user and their goal will likely irritate
them.
%
And paraphrasing Master Yoda, Roadblocks lead to anger, anger leads to
hate and hate leads to suffering.
%
  \item \textbf{Your users lose faith in your abilities to innovate}
%

\end{itemize}

\subsubsection*{How do periodic updates affect the development team?}
\begin{itemize}
  \item \textbf{Every release becomes extremely difficult}
%
Most applications of any size are complex to deploy.
%
For example in our case our entire system consists of multiple web
servers, a set of caching servers, application servers and databases.
%
To release a new version of the system we need to perform multiple
installation steps across several machines.
%
Copy the static files to one location, copy the application dlls on
all the web servers, restart the servers, poke the index page to load
the application.
%
When these steps are done manually, a human needs to make judgements
which leaves this prone to human error.
%
I'm sure you've heard of war stories where it's 2am in the morning and
your star programmer is staring blanky at a configuration file in a
production system trying to figure out how to make it work.
%
  \item \textbf{They take up developer time}
%
Developers are probably the most valuable asset of a technology
startup.
%
They are highly skilled and trained professionals and they are the
people who create the product.
%
Without a product you don't have anything.
%
It is really important for them to use their time as effectively as
possible on the things that matter the most.
%
\end{itemize}

\subsection{Flickr is a successful example of Continuous Delivery}

The folks at Flickr have perfected their deployment process to such a
degree that they even have a public page where you can see detailed
information about their latest release.
%
Let me show what it looks like right now.

If you go to the URL http://code.flickr.come/ you can find information
about their development communicty.
%
At the bottom of the page you can see 3 things: when was the last
version deployed to production, how many changes were included and how
many people produced those changes.
%
You can also see how many times the Flickr system was deployed in the
past week.
%
Last you can see the people who were involved in these deployments.

Why is this important?
%
Every business idea is a hypothesis until you get user feedback.
%
And the only way to get real user feedback is to put your product in
front of users.
%

Corrolary: don't waste money on the wrong thing.

react to the market and explore new revenue streams

Frequent release cycle, development and operations people are close to
the user.


\subsection{How do we get there?}

We need to have 2 things: continuous integration and continuous
deployment.

\subsubsection*{What is continuous integration?}
\begin{itemize}
  \item An implementation of a continuous process of that aims to
    improve the quality of software via small pieces of effort,
    applied frequently.
  \item It aims to reduce the time taken to deliver software by
    replacing the traditional practice of applying quality control
    after completing all development
\end{itemize}

For example in our project, we cosider the role of the developer to
be, to implement the feature and to write all the automated tests to
prove that this feature works.
%
In aggregate what that gives us is the ability to convince ourselves
that the entire system works every time we add new code by running all
the automated tests.
%
And since you would expect that all those tests will pass, why not
deploy the new version?



\subsubsection*{What is continuous deployment?}

\subsection{What do I need to implement a continuous integration and continuous deployment system?}

\begin{itemize}
\item \textbf{Version control}
%
What should I put in version control?
%
Everything required by your project: code, tests, database scripts,
build and deployment scripts and anything else needed to create,
install, run, and test the application.
%
In our project we use a distributed version control system called
Mercurial.
%
DVCS have several benefits over more traditional client-server systems
like Subversion.
%
The first benefit is the \textbf{ability to work offline} (e.g. network outage,
travelling)
%
When you're online there is no more waiting for slow responses from a
remote server which may be in a different country.
%
\textbf{Fast everyday operations}. Commit, rollback are instantaneous
operations and you only take the hit when you push your changesets to
the team repository or when you pull the changesets from the team
repository.
%
\textbf{Built-in redundancy} as each developer machine contains a full
clone of the repository and its history
%
It has been an ideal tool for our dristributed team and allows the
smooth integration of internal teams and external developers.
%
\item \textbf{An automated build} 
%
Whatever the mechanism, it must be possible for either a human or a
computer to run the build, test, and deployment process in an
automated fashion via the command line.
%
Additionally the steps you follow to execute the process should be
exactly the same irrespectively of whether it's a human or a computer
executing the process.
%
For our project we've chosen NAnt and TeamCity as our automated build
tools.
%
NAnt is a port of the popular Ant build tool that allows you to
automate the tedious task of building all the project artifacts.
%
It is also and XML-based scripting language that allows you to define
tasks that are part of your build process.
%
For example you can use NAnt to compile your source code, run your
tests, package your binaries for deployment, copy the installation
package on the appropriate servers.
%
TeamCity is a tool that automates the execution of our nant scripts
and collects the results in a central place and makes it centrally
available for the entire team.
%
\item \textbf{Agreement of the team} 
%
Continuous integration is a practice not a tool.
%
For every practice you need a set of ground rules and making sure
that everyone follows the rules.
%
For example one of the problems we had was commiting on a broken
pipeline.
\end{itemize}

\subsection{Our implementation of a Continuous Integration pipeline}

At a conceptual level our entire pipeline consists of 6 steps that are
automatically executed every time a developer makes any change to our
codebase: 1) archive the source code for immediate future access, 2)
compile the code, 3) run all the unit tests, 4) run automated quality
assurance tools, 5) deploy the code to our development environment,
and 6) run all the automated tests.
%
Whatever the change to our code our binaries have to go throught these
steps.

\subsubsection*{So what do all these steps do?}
\begin{itemize}
  \item \textbf{Archive source code} 
%
In this step all that we do is take a copy of the latest version of
our source code and place it in an easily accessible directory.
%
This may seem redundant when you have your version control system so
let me explain why we do it.
%
As your project size increases compilation starts to take longer and
longer.
%
Currently it takes around 18 minutes to compile the entire system on
my laptop.

  \item \textbf{Compile source code and archive binaries}
%

\end{itemize}

\subsection{How have we implemented this pipeline in TeamCity}
TeamCity has the concept of multiple build agents and a single
controller.
%
Build agents sit there idly until the controller gives them some work
to do.
%
Following a map-reduce pattern agents are triggered to do some work.
%
When that happens, they go and fetch their input from a file system,
do whatever work they need to do, and when they finish, place the
results of their processing back on a file system.
%
While working they continuously report their status back to the
contrller machine.

This sounds a bit complicated so let me give you an example of what
happens when a developer pushes a new changeset.
%
The first available agent checks out the latest version of the code
from 



\subsection{Conclusion}
About 1,5 hour after a developer pushes a new set of changes to our
version control repository, if everything goes well, we have a release
quality version of our software.
%
It is up to the business stakeholder to decide whether they actually
want to push this version up to production.

We are even able to this with an incomplete product.
%
For example in the top right corner we have finished things, but there
are.


If you want to know more about how to do this for your own projects,
the team will be here after the talk to give you some pointers.
%
All the tools we use are either open source, or have free versions for
small installations and academic institutions.


\section{Build projects around motivated individuals. Give them the environment and support they need, and trust them to get the job done}

The last agile principle I wanted to touch upon today is ``building
projects around motivated individuals. Giving them the environment and
support the need, and trust them to get the job done''.

At Orbis Access we see ourselves as a fast-growing technology startup.
%
Only two years ago in 2010, the entire team consisted of 4 people;
Alvin, myself and two developers in Bermuda.
%
Over the past two years, the team has grown to 36 people today and we
expect to reach 55 by the end of 2012.

We go to great pain to find great people, we hire people from the best
Universities from around the world including from Imperial.
%
For example Qi and I were students here at Imperial not so long ago.
%
Say more about the specific courses Qi and I did.

As you can see the members of our team come from more than 15
countries which creates a very diverse working environment.
%
Being able to view situations from different perspectives leads to
greater creativity, innovation and improves our problem solving
ability.




\end{document}
