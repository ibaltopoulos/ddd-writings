\documentclass{article}

\begin{document}
\title{Agile development in a financial services startup: How ``code first''
helps us release 15 times a day}
\date{October 25, 2011}

\maketitle

\begin{abstract}
Orbis is a global investment firm managing USD25B for institutional
clients and high net-worth investors.
%
Our vision is to 'democratize' high quality investments by offering
them direct to everyday investors, through our new online platform
OrbisAccess.
%
In this talk we go over our journey so far to build a high-tech
startup from the ground up.
%
We reflect on how domain driven design (DDD) and ``code first'' have
helped us keep our core model clean from technology distractions,
create testable code, and remain flexible in light of business
requirement changes.
%
Finally we share our experience using continuous delivery to quickly
evolve functionality, reduce the risk of deploying to production, and
get real-time feedback on the progress of the entire project.
\end{abstract}




\paragraph{Speaker bio} 
Alvin Mok is the global manager for OrbisAccess. 
%
His current focus is to incubate an internal start-up which will
revolutionize the financial industry with state-of-the-art technology,
novel business model and innovative products.
%
Prior to joining Orbis, Alvin was an Engagement Manager at McKinsey
Hong Kong, where he was actively involved in the high tech and
infrastructure practices as well as co-founded the Greater China real
estate practice.
%
Earlier in his career, Alvin was also a venture capitalist, held
positions in Microsoft, and founded his own startup.
%
Alvin holds a BASc from University of Toronto where he received the
Governor General Award in the Engineering Science program and a MBA
from Harvard Business School where he was a Baker Scholar.

\paragraph{Speaker bio}
Ioannis Baltopoulos is the back-end developer lead for Orbis Access.
%
Prior to Orbis, Ioannis was studying for his PhD degree at the
University of Cambridge.
%
He has worked at Microsoft Research Cambridge, Sun Microsystems
Research Labs in Silicon Valley and CERN in Geneva on a variety of
areas ranging from web services for distributed computing to type
systems in multi-tier language compilers.
%
Ioannis holds a BSc degree from the University of Kent in Computer
Science and a MSc degree in Advanced Computing from Imperial College.

\paragraph{Speaker bio}
Paul Coleman is a software developer for Orbis Investment Management
Limited.  He is responsible for parts of the back end, business logic
and data layers in the OrbisAccess system.
%
Prior to Orbis, Paul worked for Logic Communications as a Vice President
of Marketing, responsible for the market strategy for all of Logic's
Business and Residential products and services.
%
Paul also worked for Microsoft and was responsible for the user
interface implementation of Office, technical and international
competition strategy, and delivering technical presentations to
executives from around the globe.
%
Paul holds an MBA degree from the University of Washington and a
computer science undergraduate degree with honors from Cornell
University.  Paul is also a technology advisor to the Bermuda
Government and Vice President of Bermuda's .NET User Group.
%
Paul is Bermudian and lives in Bermuda with his wife and two young children.

\paragraph{Speaker bio} 
Qi Chen works as a back-end developer in the OrbisAccess London team
where he's responsible for business logic implementation and database
deployment automation.
%
He graduated with a first class MEng Computing degree from Imperial
College.
%
Prior to that, he did a half year placement for IBM in their network
management software development team where he was responsible for
software accessibility testing and management, automatic deploying and
testing framework development, as well as assisting project
management.

\end{document}

\newpage

\section{Structure}
\subsection{Intro (10 mins, Paul?)}
I think Orbis has an exceptionally good back story that fits perfectly with his first and second points.  Additionally it sets us apart from the rest of the financial services industry.  It's the Allan Gray story as motivation for why we are doing what we are doing.  The crux of the story is that he's had the vision and inclination for a long time but for retail to be successful the cost of business needs to be lower and we are using technology to do so.

The only downside I see about this as the opening / motivation is that planning for retirement is probably not on the minds of the students.  However we can draw them in by asking about their parents, and if they plan to live long enough to get to their parent's age

\subsection{OA Operations Areas (10 mins, Alvin?)}
The motivation sets up what we are doing and why IT is the single most important tool for us winning in the retail space.  Then we move into all the operational aspects of the business - enumerating marketing, information gathering by customers, account opening, transactions, funds, customer support.  And at a high level at this point we can discuss how we are using technology to make the operational area better for the customer and cheaper for us.

\subsection{Drill down (20 mins, Ioannis, Qi)}
Finally we can drill into one or two of these areas to explain in detail what technology choices we've made and why and pros, the cons, the impact to schedule, the skills on the team we need to be successful in that area etc.  What areas we need to drill down on would still be tbd.

\subsection{Closing (5 mins, ?)}

\subsection{Questions (12 mins, Ioannis, Qi, Paul, Alvin)}

\subsection{Goodbye (3 mins, Alvin)}



\section{Points}


You're part of this startup, you've got a vision but you don't know the details of what you're implementing until 

\paragraph{Starting development from the database model is inefficient}
\begin{itemize} 
\item for every new concept/table you introduce, you have to create
  data access objects in the object language

\item it gives higher priority to data organization rather than data
  usage
\item your persistence mechanism imposes a relational way of viewing
  the world. For some applications the relational model is not
  appropriate. A hierarchical representation would have been better

\item you don't even know the domain yet, we're discovering it as we
  go along
\item during the ramp up stage of the project there are frequent
  changes to the model. That means constant code churn
\end{itemize}

