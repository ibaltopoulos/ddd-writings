\documentclass{article}

\begin{document}
\title{Architectural principles}

\paragraph{The entity model should be independent of all other modules.}

\begin{itemize}
  \item You cannot import the data transfer objects (DTOs). This
    implies that when you want to report a error value back to the
    front end your best solution is to declare a domain specific
    ModelError and throw an exception.
  \item Logic that is specific to an entity should live with the
    entity unless it requires multiple different entities, in which
    case this logic should live inside the domain services.
  \item The entity model contains the repository interfaces but it
    should not contain the repository interfaces. The implementations
    are provided dynamically at runtime and can be substituted with
    whatever is appropriate (in-memory database instances, db4o, sql
    server, etc).
  \item Entity classes should not be split between data and logic
    files.  The entire definition of a class should be in a single
    file unless we are using code generation. This keeps everything
    localized. To ensure consistency we should have regions in the
    entity files. One region for the data definition and another
    region for the behaviour/logic.
  \item The entity model should not reference the external services.
    These services have access to the wcf stack and are subject to
    change. The recommended way is to either return the required
    information for the cross-domain call as return values in domain
    service interfaces or, alternatively, as C\# events.
   \item The entities that manipulate status codes and states should
     implement the state pattern.
   \item You are allowed to create value objects that are used for
     communicating values between the entity model and the outside
     callers. These value objects are the contract with the enternal
     world and don't need to be persisted.
   \item For persistence reasons entities should ALWAYS have a
     protected no-arg constructor in addition to any other constructor
     that they require for their own initialization.
   \item All the properties of the entities need to be declared with a
     virtual modifier and the set method needs to be protected. The
     getter can be public.
   \item The non-default entity constructor needs to be declared as
     internal. This ensures that only the entity model project can
     create entity instances. Of course you can abuse it with
     reflection but it should be good enough for most cases.
   \item Entity instantiation should happen through static factory
     classes when instantiation is complex or alternatively through
     factory methods.

\end{itemize}


The domain services should contain the logic for performing particular

\end{document}